\documentclass[sigplan,10pt,noacm]{acmart}
\settopmatter{printfolios=false,printccs=false,printacmref=false}

\acmConference{A Research Agenda for Formal Methods in the Netherlands}%
  {September 3--4}%
  {Lorentz Center, Leiden, The Netherlands}
\acmYear{2018}
\acmISBN{} % \acmISBN{978-x-xxxx-xxxx-x/YY/MM}
\acmDOI{} % \acmDOI{10.1145/nnnnnnn.nnnnnnn}
\startPage{1}

\setcopyright{none}

\bibliographystyle{ACM-Reference-Format}
\citestyle{acmnumeric}     %% For numeric citations

\begin{document}

\title{Formal Methods for Security?}

\author{Erik Poll}
\affiliation{
  \department{Digital Security group}        
  \institution{Radboud University}          
}
\email{erikpoll@cs.ru.nl}          

\maketitle
\pagestyle{empty}


This position paper sketches some 
opportunities in applying formal methods for security,
more specifically security of software.
%There are two levels at which we can try to use formal methods
%here: at the level of an individual application, to ensure
%meeting application-specific security requirments, or at the
%level of the programming language, to make the language or
%its use `safer'.
% and less prone to generic security vulnerabilities.

\subsection*{A killer application for formal methods?}

At first sight  -- and even at second sight -- security looks like
an interesting application area for formal methods. 

One reason for this is that security flaws have a higher impact
than more harmless bugs. This might justify or even require the extra
effort to invest in using formal methods.  Indeed, the highest
levels of certification using the Common Criteria security
evaluation standard require the use of formal methods.

Another reason is that some security problems are orthogonal to
-- or at least largely independent of -- the functionality of a
system.  Indeed, security vulnerabilities can be seen as
`anti-functionality': functionality that is unintentionally
provided, also to attackers, which should not be available at
all.  Given that writing complete functional specifications is
hard, and totally unfeasible in most cases, concentrating on
partial specs that ensure some generic safety properties 
problems might provide a better return of investment, also
because such specs could be re-used across applications.

Unfortunately, things are not so simple.  Security properties can
be tricky to specify. Indeed, attackers can be very creative in
finding and exploiting new loopholes. A common and natural way to
specify security properties is in a `negative' way, by saying
that something -- some type of attack -- is not supposed to be
possible. For example, a web application should not be vulnerable
to SQL injection or XSS. List of these negative properties are
useful in testing, as they suggest negative tests (i.e.\
test-cases which are supposed to fail, by triggering some error
response), but they are not immediately helpful in construction.
Moreover, these lists of negative properties are typically
incomplete. They only address a limited set of known potential
problems, not all potential problems.

Moreover, sometimes security problems arise because it turns
out fundamental assumptions about programs can be broken by an
attacker, 
invalidating the very abstractions that we use in formal methods
to reason about programs.  Classic examples here are fault
attacks (for example the Rowhammer attack to flip some 
bits in DRAM memory) or information leaking through side-channels
(for example the Spectre and Meltdown timing attacks on modern
CPUs). Of course, our formal models could be refined to accomodate
these low-level attacks, but at the (high) cost of extra
complexity.

\subsection*{Security functionality $\not=$ secure functionality}

When investigating at the security of software, it is natural to
focus on the security functionality, i.e.\ the functionality
specifically intended to provide some security guarantees, such
as access control checks or security protocols like SSL/TLS.
Such functionality is obviously security-critical.

However, while this may suggest rewarding aspects or components
to investigate, it is dangerous to fall into the trap of thinking
that such security functionality is the only or even the most
important area to look for problems.  Not only the security
functionality has to be secure: \emph{all} functionality needs to
be, as security vulnerabilities can lurk in any line of code that
can be triggered by input controlled by the attacker.  The bulk of
security bugs is not in code specifically aimed at achieving some
security goal, but is in more mundane functionality, say the
parsing of PDF files or the rendering of some graphical
format, with Flash as the most notorious example.

\subsection*{LangSec: back to basics?}

The paradigm of LangSec (language-theoretic security)\footnote{See also \url{http:\\langsec.org}, esp. \cite{LangSecBoF}, or
\cite{Momot2016} for a more recent entry point into the LangSec literature.}
provides very good insights into the root causes of the overwhelming
majority of security flaws, namely bugs in handling input, which
typically boil down to bugs in the parsing and
processing input languages and formats, rather than bugs in
the application logic.

One problem is with the input languages themselves: they are
typically overly complex, too expressive, and poorly -- and
informally -- specified.  To make matters worse, there are many
of these languages, at every level of the network and software
stack, and they can be combined or nested. A further problem here
is that the code to process these languages is typically
hand-coded, and not obtained using parser generation.

Ironically -- or embarrassingly, for the computing science
community -- theories for formal language definitions and parser
generation are some of the oldest and most established areas
in formal methods. Still, somehow the whole world is still
writing long prose documents to specify languages and protocols,
and then hand-coding parsers -- often in memory unsafe
languages like C or C++, where the potential security impact of
flaws is the biggest (namely remote code execution).
There is a huge opportunity here to provide better notations and tools
to prevent all this misery. Or maybe these already exist, and we
should do a better job in training people -- incl.\ our students
-- on how to use them?

One step further from formal specs and associated code generation
for parsers and pretty printers would be domain-specific
programming languages to support different input formats and
languages as first class citizens, as envisioned in Wyvern
\cite{Wyvern}.

\subsection*{Security Testing \& Model Extraction}

The past decade has seen a lot of fruitful interaction between
formal methods and testing, also in security testing.  An
interesting trend is the use of formal methods, notably
symbolic/concolic execution, for security testing
\cite{SAGE,driller} or even going one step further and actually
develop exploits as in the angr tool \cite{angr}.  If we
cannot get developers to use formal methods, maybe we should
concentrate efforts on getting security testers and hackers to use
formal methods?
(Of course, with more robust parser code that has generated from
formal language specs, as we argue for above, it should be harder
to find security flaws \ldots)

Test techniques can also provide a way to obtain formal specs
from implementations.  Given the difficulty of obtaining formal
models this is an interesting direction of work.  Existing
fuzzers can already reverse-engineer input formats
\cite{Polyglot,Prospex}, and state machine inference can be used
to extract security-relevant behaviour from code
\cite{LangSec2015}.

All such formal techniques for security testing or model inference
could be combined with machine-based learning or AI approaches,
to improve results and/or the level of automation. 

\subsection*{Practical information flow}

Information flow properties are an interesting class of security
properties. Information flow can be used to track potential
leakage of confidential information or to track the flow of
tainted input to places where such input may do damage.  Research
on information flow has a long history, dating back to the 1970s
\cite{Denning77}, and ad-hoc information analyses are implemented
in code analysis for security flaws -- aka Static Application
Security Testing (SAST), by tools such as Coverity, Checkmarx, or
Fortify, but flexible and practical approaches to express and
enforce information flow for programs in popular programming
languages (e.g.\ \cite{Dietl2011} for Java) are still rare and
not commonly used. 

\subsection*{New (and safer) programming languages, new opportunities?}

One positive development for security in recent years has been
the advent of new programming languages -- Rust, D, Go, Swift,
Nim, \ldots -- where safety is very much a design goal. Some of
these languages are specifically aimed for low-level programming
and might become viable, widely-used, and safer alternatives
for C/C++ and then reduce the prevalence of memory corruption
problems.

The advent of these new languages is a double-edged sword.  An
advantage is that they are designed to be more amenable to formal
analysis and have some security guarantees built in at the
language level. A downside is that new languages require new
tools. Building and maintaining good formal methods tools is a
major bottleneck, so here the advent of new languages is bad
news.  For researchers these new languages represent new research
opportunities. This may be good news, if this new research gets
us further, or bad news, if this research is merely repeating and
recycling the same old ideas without getting us further.

%\bibliography{researchagenda_fm.bib}

%%% -*-BibTeX-*-
%%% Do NOT edit. File created by BibTeX with style
%%% ACM-Reference-Format-Journals [18-Jan-2012].

\begin{thebibliography}{11}

%%% ====================================================================
%%% NOTE TO THE USER: you can override these defaults by providing
%%% customized versions of any of these macros before the \bibliography
%%% command.  Each of them MUST provide its own final punctuation,
%%% except for \shownote{}, \showDOI{}, and \showURL{}.  The latter two
%%% do not use final punctuation, in order to avoid confusing it with
%%% the Web address.
%%%
%%% To suppress output of a particular field, define its macro to expand
%%% to an empty string, or better, \unskip, like this:
%%%
%%% \newcommand{\showDOI}[1]{\unskip}   % LaTeX syntax
%%%
%%% \def \showDOI #1{\unskip}           % plain TeX syntax
%%%
%%% ====================================================================

\ifx \showCODEN    \undefined \def \showCODEN     #1{\unskip}     \fi
\ifx \showDOI      \undefined \def \showDOI       #1{#1}\fi
\ifx \showISBNx    \undefined \def \showISBNx     #1{\unskip}     \fi
\ifx \showISBNxiii \undefined \def \showISBNxiii  #1{\unskip}     \fi
\ifx \showISSN     \undefined \def \showISSN      #1{\unskip}     \fi
\ifx \showLCCN     \undefined \def \showLCCN      #1{\unskip}     \fi
\ifx \shownote     \undefined \def \shownote      #1{#1}          \fi
\ifx \showarticletitle \undefined \def \showarticletitle #1{#1}   \fi
\ifx \showURL      \undefined \def \showURL       {\relax}        \fi
% The following commands are used for tagged output and should be
% invisible to TeX
\providecommand\bibfield[2]{#2}
\providecommand\bibinfo[2]{#2}
\providecommand\natexlab[1]{#1}
\providecommand\showeprint[2][]{arXiv:#2}

\bibitem[\protect\citeauthoryear{??}{Lan}{2013}]%
        {LangSecBoF}
 \bibinfo{year}{2013}\natexlab{}.
\newblock \bibinfo{title}{{LangSec}: Recognition, Validation, and Compositional
  Correctness for Real World Security}.
\newblock   (\bibinfo{year}{2013}).
\newblock
\newblock
\shownote{USENIX Security BoF hand-out. Available from
  \url{http://langsec.org/bof-handout.pdf}.}


\bibitem[\protect\citeauthoryear{Caballero, Yin, Liang, and Song}{Caballero
  et~al\mbox{.}}{2007}]%
        {Polyglot}
\bibfield{author}{\bibinfo{person}{J. Caballero}, \bibinfo{person}{H. Yin},
  \bibinfo{person}{Z. Liang}, {and} \bibinfo{person}{D. Song}.}
  \bibinfo{year}{2007}\natexlab{}.
\newblock \showarticletitle{Polyglot: Automatic extraction of protocol message
  format using dynamic binary analysis}. In \bibinfo{booktitle}{\emph{CCS'07}}.
  ACM, \bibinfo{pages}{317--329}.
\newblock


\bibitem[\protect\citeauthoryear{Comparetti, Wondracek, Kruegel, and
  Kirda}{Comparetti et~al\mbox{.}}{2009}]%
        {Prospex}
\bibfield{author}{\bibinfo{person}{P.M. Comparetti}, \bibinfo{person}{G.
  Wondracek}, \bibinfo{person}{C. Kruegel}, {and} \bibinfo{person}{E. Kirda}.}
  \bibinfo{year}{2009}\natexlab{}.
\newblock \showarticletitle{Prospex: Protocol specification extraction}. In
  \bibinfo{booktitle}{\emph{Security and Privacy, 2009 30th IEEE Symposium
  on}}. IEEE, \bibinfo{pages}{110--125}.
\newblock


\bibitem[\protect\citeauthoryear{Denning and Denning}{Denning and
  Denning}{1977}]%
        {Denning77}
\bibfield{author}{\bibinfo{person}{D.E. Denning} {and} \bibinfo{person}{P.J.
  Denning}.} \bibinfo{year}{1977}\natexlab{}.
\newblock \showarticletitle{Certification of Programs for Secure Information
  Flow}.
\newblock \bibinfo{journal}{\emph{Commun. ACM}} \bibinfo{volume}{20},
  \bibinfo{number}{7} (\bibinfo{date}{July} \bibinfo{year}{1977}),
  \bibinfo{pages}{504--513}.
\newblock


\bibitem[\protect\citeauthoryear{Dietl, Dietzel, Ernst, Mu\c{s}lu, and
  Schiller}{Dietl et~al\mbox{.}}{2011}]%
        {Dietl2011}
\bibfield{author}{\bibinfo{person}{W. Dietl}, \bibinfo{person}{S. Dietzel},
  \bibinfo{person}{M.D. Ernst}, \bibinfo{person}{K. Mu\c{s}lu}, {and}
  \bibinfo{person}{T.W. Schiller}.} \bibinfo{year}{2011}\natexlab{}.
\newblock \showarticletitle{Building and Using Pluggable Type-checkers}. In
  \bibinfo{booktitle}{\emph{ICSE'11}}. \bibinfo{publisher}{ACM},
  \bibinfo{pages}{681--690}.
\newblock


\bibitem[\protect\citeauthoryear{Godefroid, Levin, and Molnar}{Godefroid
  et~al\mbox{.}}{2012}]%
        {SAGE}
\bibfield{author}{\bibinfo{person}{P. Godefroid}, \bibinfo{person}{M.Y. Levin},
  {and} \bibinfo{person}{D. Molnar}.} \bibinfo{year}{2012}\natexlab{}.
\newblock \showarticletitle{SAGE: whitebox fuzzing for security testing}.
\newblock \bibinfo{journal}{\emph{Queue}} \bibinfo{volume}{10},
  \bibinfo{number}{1} (\bibinfo{year}{2012}), \bibinfo{pages}{20}.
\newblock


\bibitem[\protect\citeauthoryear{Momot, Bratus, Hallberg, and Patterson}{Momot
  et~al\mbox{.}}{2016}]%
        {Momot2016}
\bibfield{author}{\bibinfo{person}{F. Momot}, \bibinfo{person}{S. Bratus},
  \bibinfo{person}{S.M. Hallberg}, {and} \bibinfo{person}{M.L. Patterson}.}
  \bibinfo{year}{2016}\natexlab{}.
\newblock \showarticletitle{The seven turrets of {Babel}: A taxonomy of
  {LangSec} errors and how to expunge them}. In
  \bibinfo{booktitle}{\emph{Cybersecurity Development (SecDev)}}. IEEE,
  \bibinfo{pages}{45--52}.
\newblock


\bibitem[\protect\citeauthoryear{Omar, Kurilova, Nistor, Chung, Potanin, and
  Aldrich}{Omar et~al\mbox{.}}{2014}]%
        {Wyvern}
\bibfield{author}{\bibinfo{person}{C. Omar}, \bibinfo{person}{D. Kurilova},
  \bibinfo{person}{L. Nistor}, \bibinfo{person}{B. Chung}, \bibinfo{person}{A.
  Potanin}, {and} \bibinfo{person}{J. Aldrich}.}
  \bibinfo{year}{2014}\natexlab{}.
\newblock \showarticletitle{Safely composable type-specific languages}. In
  \bibinfo{booktitle}{\emph{ECOOP'14}} \emph{(\bibinfo{series}{LNCS})},
  Vol.~\bibinfo{volume}{8586}. Springer, \bibinfo{pages}{105--130}.
\newblock


\bibitem[\protect\citeauthoryear{Poll, de~Ruiter, and Schubert}{Poll
  et~al\mbox{.}}{2015}]%
        {LangSec2015}
\bibfield{author}{\bibinfo{person}{E. Poll}, \bibinfo{person}{J. de Ruiter},
  {and} \bibinfo{person}{A. Schubert}.} \bibinfo{year}{2015}\natexlab{}.
\newblock \showarticletitle{Protocol state machines and session languages:
  specification, implementation, and security flaws}. In
  \bibinfo{booktitle}{\emph{Workshop on Language-Theoretic Security
  (LangSec'15), Symposium on Security and Privacy Workshops}}.
  \bibinfo{publisher}{IEEE}, \bibinfo{pages}{125 -- 133}.
\newblock


\bibitem[\protect\citeauthoryear{Shoshitaishvili, Wang, Salls, Stephens,
  Polino, Dutcher, Grosen, Feng, Hauser, Kruegel, and Vigna}{Shoshitaishvili
  et~al\mbox{.}}{2016}]%
        {angr}
\bibfield{author}{\bibinfo{person}{Y. Shoshitaishvili}, \bibinfo{person}{R.
  Wang}, \bibinfo{person}{C. Salls}, \bibinfo{person}{N. Stephens},
  \bibinfo{person}{M. Polino}, \bibinfo{person}{A. Dutcher},
  \bibinfo{person}{J. Grosen}, \bibinfo{person}{S. Feng}, \bibinfo{person}{C.
  Hauser}, \bibinfo{person}{C. Kruegel}, {and} \bibinfo{person}{G. Vigna}.}
  \bibinfo{year}{2016}\natexlab{}.
\newblock \showarticletitle{{SOK}:(state of) the art of war: Offensive
  techniques in binary analysis}. In \bibinfo{booktitle}{\emph{Symposium on
  Security and Privacy (SP)}}. IEEE, \bibinfo{pages}{138--157}.
\newblock


\bibitem[\protect\citeauthoryear{Stephens, Grosen, Salls, Dutcher, Wang,
  Corbetta, Shoshitaishvili, Kruegel, and Vigna}{Stephens
  et~al\mbox{.}}{2016}]%
        {driller}
\bibfield{author}{\bibinfo{person}{N. Stephens}, \bibinfo{person}{J. Grosen},
  \bibinfo{person}{C. Salls}, \bibinfo{person}{A. Dutcher}, \bibinfo{person}{R.
  Wang}, \bibinfo{person}{J. Corbetta}, \bibinfo{person}{Y. Shoshitaishvili},
  \bibinfo{person}{C. Kruegel}, {and} \bibinfo{person}{G. Vigna}.}
  \bibinfo{year}{2016}\natexlab{}.
\newblock \showarticletitle{Driller: Augmenting Fuzzing Through Selective
  Symbolic Execution}. In \bibinfo{booktitle}{\emph{NDSS'16}},
  Vol.~\bibinfo{volume}{16}. \bibinfo{publisher}{Internet Society},
  \bibinfo{pages}{1--16}.
\newblock


\end{thebibliography}
\end{document}
