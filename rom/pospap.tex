%% For double-blind review submission, w/o CCS and ACM Reference (max submission space)
\documentclass[sigplan,10pt]{acmart}\settopmatter{printfolios=true,printccs=false,printacmref=false}
%% For double-blind review submission, w/ CCS and ACM Reference
%\documentclass[sigplan,10pt,review,anonymous]{acmart}\settopmatter{printfolios=true}
%% For single-blind review submission, w/o CCS and ACM Reference (max submission space)
%\documentclass[sigplan,10pt,review]{acmart}\settopmatter{printfolios=true,printccs=false,printacmref=false}
%% For single-blind review submission, w/ CCS and ACM Reference
%\documentclass[sigplan,10pt,review]{acmart}\settopmatter{printfolios=true}
%% For final camera-ready submission, w/ required CCS and ACM Reference
%\documentclass[sigplan,10pt]{acmart}\settopmatter{}


%% Conference information
%% Supplied to authors by publisher for camera-ready submission;
%% use defaults for review submission.
\acmConference[Lorentz]{}{September 03--04, 2018}{Leiden, The Netherlands}
\acmYear{2018}
\acmISBN{} % \acmISBN{978-x-xxxx-xxxx-x/YY/MM}
\acmDOI{} % \acmDOI{10.1145/nnnnnnn.nnnnnnn}
\startPage{1}

%% Copyright information
%% Supplied to authors (based on authors' rights management selection;
%% see authors.acm.org) by publisher for camera-ready submission;
%% use 'none' for review submission.
\setcopyright{none}
%\setcopyright{acmcopyright}
%\setcopyright{acmlicensed}
%\setcopyright{rightsretained}
%\copyrightyear{2017}           %% If different from \acmYear

%% Bibliography style
\bibliographystyle{ACM-Reference-Format}
%% Citation style
\citestyle{acmauthoryear}  %% For author/year citations
%\citestyle{acmnumeric}     %% For numeric citations
%\setcitestyle{nosort}      %% With 'acmnumeric', to disable automatic
                            %% sorting of references within a single citation;
                            %% e.g., \cite{Smith99,Carpenter05,Baker12}
                            %% rendered as [14,5,2] rather than [2,5,14].
%\setcitesyle{nocompress}   %% With 'acmnumeric', to disable automatic
                            %% compression of sequential references within a
                            %% single citation;
                            %% e.g., \cite{Baker12,Baker14,Baker16}
                            %% rendered as [2,3,4] rather than [2-4].



\begin{document}

\title[]{Model checking in biology and health care} 


\author{Rom Langerak}
%\authornote{with author1 note}          %% \authornote is optional;
                                        %% can be repeated if necessary
%\orcid{nnnn-nnnn-nnnn-nnnn}             %% \orcid is optional
\affiliation{
%  \position{Position1}
  \department{Formal Methods and Tools}           %% \department is recommended
  \institution{University of Twente}            %% \institution is required
%  \streetaddress{Street1 Address1}
%  \city{City1}
%  \state{State1}
%  \postcode{Post-Code1}
%  \country{Country1}                    %% \country is recommended
}
\email{r.langerak@utwente.nl}          %% \email is recommended


\begin{abstract}

Model checking is a useful technique for analyzing models in biology and
health care.
Here we point out several research topics concerning the modeling of
various networks in biology, and modeling
of diagnostic and treatment protocols in health care.

\end{abstract}


\maketitle


\section{Introduction}
Many complex biological phenomena can be modeled as networks. Prominent
examples within biological cells are metabolic neworks, signaling networks, and
gene regulatory networks. Understanding the quantitative and qualitative
behavior of such networks is an important prerequisite for curing various
diseases. Modeling these networks within a framework that allows model checking
is an attractive way of gaining such understanding \cite{Brim:2013aa,David2015}.

In Twente the contribution of computer science to this research has
concentrated on the use of
UPPAAL \cite{UPPAAL:aa}. UPPAAL is attractive as it is a mature tool allowing
a compositional approach, and with a graphical interface that facilitates
communication with non-experts in formal methods.

Kinase signaling networks have been modeled 
in the context of osteoarthritis \cite{Scholma:2013aa,Scholma:2014aa}. A tool
called ANIMO has been created \cite{ANIMO:aa} that makes use of UPPAAL and is
intended to be used by molecular biologists. ANIMO has been succesfully used to
model realistically sized biological networks 
\cite{Schivo:2012aa,Schivo:2013ab,Schivo:2014aa,Scholma:2014ab,Schivo:2016aa}.
Below we list several research topics for the modeling and analysis of
biological networks.

Another line of research is modeling protocols for diagnosis and treatment, and
then using model checking for analysing these protocols for effectiveness,
efficiency,  and costs. In Twente such analysis has been performed using UPPAAL
for prostrate cancer \cite{Schivo:2015,Degeling:2017} and tooth wear
\cite{Rooijen:2018,Choudry:2018} (in collaboration with the Academic Centre for
Dentistry Amsterdam). Below we list several research topic for  the modeling
and analysis of diagnosis and treatment protocols.

\section{Research topics for modeling and analysis of biological networks}


\begin{description}
\item[Tools] Modeling and analysis techniques should be offered to biologists
and medical researchers via tools that hide as much as possible the technical
complexities of the underlying computer science models. One way of achieving
this is to try to stick as close as possible to user interfaces of the mostly
informal network
tools that have been developed already in biology.

\item[Model generation] Creating a model can be a time consuming task. Often
the information needed for constructing a model can be found in literature or
databases.  Therefore we need to investigate techniques for building initial
models from information that is extracted from heterogeneous sources. It is
important to be able to deal with missing or incomplete information.

\item[Parameter fitting and sensitivity] Much of the effort in creating a model
goes into providing the right parameters for a model. This is related to the
issue of parameter sensitivity and robustness. Especially for parameters that
vary from patient to patient is is important to establish that model properties
are not critically dependent on such parameters.

\item[Simulation and visualization] Simulations are needed initially to gain
confidence in the correctness of a model. Once confidence has been gained in
the correctness, a model can be explored by simulating it with various stimuli.
In this way hypotheses can be checked by in-silico experimentations. Challenges
are the efficiency of such simulations, especcially if reactions may take place
at different timescales. In addition, the way results are visually and
graphically presented to
researchers of medical practitioners  is of crucial importance.

\item[Relating models and experiments] Experiments will always form an
indispensible component of biological research, and modeling can geatly enhance
the effectivity and efficiency of experiments. Models can be of great help0 in
suggesting experiments and thereby pruning the large amount of possible
experiments. Models are also important in interpretating the (often verly
large) amount of experimental data - just indicating which data is in
accordance with the model sofar, and which data is not, is already extremely
useful. And more research should be performed on automatic suggestions for
model improvement in the light of new experimentalo data.

\item[Using model checking for drug synthesis] For some goal in a network model,
model
checking can provide the stimuli that have to be offered to a network in order
to reach that goal, by analyzing the trace leading to the goal.
In this way model checking is a powerful technique supporting drug synthesis.
Since network
models may have an enormous state space, the challenge is to find abstraction
and high performance computing technqiues that enable to check large scale
network models.


\end{description}

\section{Research topics for modeling and analysis of treatment protocols}


\begin{description}
\item[Tools] The ambition is to create a tool that enables health practitioners
to create their own treatment protocols, and analyze them. This asks for a
domain specific language that is both easy to use and sufficiently flexible and
expressive to deal with many different scenarios.

\item[Educated guesses for parameters] Usually when modeling a treatment
protocol many parameters are unknown or not known precisely, and it would be
too costly or time consuming to establish such parameters by clinical trials.
What is needed is a framework to deal with such educated guesses; by sensitivity
analysis or parameter sweeps it could be established which parameters need to
be established with more precision, and which parameters are not so crucial for
the analysis outcomes.


\item[Optimization] It would be very useful to establish the optimal protocol
under
some constraint. What is the most effective protocol given a certain budget, or
what is the most economic protocol that is able to obtain a given level of
effectiveness? This asks for a theory of optimization of timed stochastic
processes with costs.

\end{description}


\bibliography{References}



\end{document}
