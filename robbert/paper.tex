%% For double-blind review submission
\documentclass[sigplan]{acmart}
\settopmatter{printfolios=true,printccs=false,printacmref=false}
\renewcommand\footnotetextcopyrightpermission[1]{} % removes footnote with conference information in first column
\fancyfoot[RO,LE]{} % Remove "Proceedings of the ACM on Programming Languages"
%% For single-blind review submission
%\documentclass[acmlarge,review]{acmart}\settopmatter{printfolios=true}
%% For final camera-ready submission
%\documentclass[acmsmall,screen,nocopyrightspace]{acmart}
%\settopmatter{printfolios=true,printccs=false,printacmref=false}

\usepackage{amsmath}
\usepackage{xspace}
\usepackage{flushend}

%% Some recommended packages.
\usepackage{booktabs}   %% For formal tables:
                        %% http://ctan.org/pkg/booktabs
\usepackage{subcaption} %% For complex figures with subfigures/subcaptions
                        %% http://ctan.org/pkg/subcaption


\newcommand{\ie}{\textit{i.e.,}\xspace}
\newcommand{\eg}{\textit{e.g.,}\xspace}
\newcommand{\etc}{\textit{etc.}\xspace}
\newcommand{\me}[1]{#1}

\begin{document}

\title{Sound and modular formal methods}

\fancypagestyle{firstpagestyle}{
  \fancyhf{} \fancyfoot[RO,LE]{}
}

\author{Robbert Krebbers}
	\affiliation{%
    \institution{Delft University of Technology}
		\city{Delft}
    \country{The Netherlands}
  }

\begin{abstract}
This short paper is a contribution to the Lorentz Center workshop on ``A Research
Agenda for Formal Methods in The Netherlands'' on September 3 and 4, 2018 in
Leiden.
%In this paper, I describe my research agenda on formal methods, as well as my
%vision on future formal methods research.
\end{abstract}

\maketitle

\section{Introduction}

Depending on the way software is used, and the resources that are available
to ensure its quality, it is desired to establish different correctness properties.
In some cases, safety properties may be sufficient,
%(\ie to ensure that software does not crash)
whereas in other cases (in particular, mission critical software), it would
be desired to establish full functional correctness.
%(\ie that the behavior of software corresponds to a detailed specification
%written in some formalism).

As such, I believe it is important that the formal methods community in
the Netherlands develops a large variety of formal methods techniques and
tools---ranging from static analysis, type systems and model checkers, to
methods for establishing full functional correctness.
This way, we can make sure that we---as the Dutch formal methods community---cover the whole spectrum.

Regardless of the actual tools or techniques that are being used, it is
crucial that these are \emph{sound} and \emph{modular}.
Soundness means that whenever the tool says that software satisfies a certain
property, that is undoubtedly the case.
After all, we do not want to give false pretenses---software that is
said to be formally verified should not go wrong.
To achieve this, we should use sound mathematical principles as the basis to
build tools and develop techniques.
On top of that, we should make sure that the implementations of the tools we are
using are mathematically verified.

Modularity means that each component of a piece of software can be verified
in isolation, and that when all of the components are combined into a larger
piece of software, we obtain the intended correctness property for the
software as a whole.
Modularity is crucial to achieve scalability, which is crucial
to use formal methods in the large.

\section{Current research}
My research has focused on the following topics:

\begin{enumerate}
\item How to develop formal semantics of realistic programming languages (like
  C and Rust), and use said formal semantics to reason about such languages.
\item How to develop \emph{separation logics} and \emph{logical relations}
  that support modular reasoning
  about daring programming features such as
  fine-grained concurrency, higher-order functions, non-local control, \etc
\item How to modularly establish meta theoretical properties of \emph{whole} programming
  languages, like type safety.
\item How to develop tools that enable convenient and proved sound reasoning
  about programs.
\end{enumerate}

As the basis of my research I am using proof assistants, which can be used
to specify programming languages, to reason about programs, and
to validate mathematical results, in the most reliable and trustworthy way.
I am an active user of the Coq proof assistant~\cite{coq} and nearly all of
my recent research has been entirely formalized using it.

Below I will list some noteworthy research projects that I have been involved
in:
\begin{itemize}
\item As part of my PhD~\cite{nonlocal,expressions,phdthesis}, I have developed a formal
  semantics of a large part of the C programming language, based on the
  official specification of C from the C11 standard.
  My C semantics, called \textbf{CH\(_2\)O}, comes in the form of a type system, an operational and
  executable semantics, and a separation logic.
  All of these components have been defined in Coq and are proved to match up
  with each other.
\item I am one of the main developers of
  \textbf{Iris}~\cite{iris2,iris3,iris_ground_up}---a framework for higher-order
  concurrent separation logic, which has been implemented in the Coq proof
  assistant and deployed very effectively in a wide variety of verification
  projects world-wide.
  Among many other things, we have used Iris to establish the correctness of the
  Rust type system and some of its standard libraries~\cite{rustbelt}, and to
  develop an expressive logic to reason about refinements of concurrent programs~\cite{reloc}.
\item I have worked on so-called \emph{tactic} languages for carrying out proofs in a proof assistant.
  Notably, I have co-developed the tactic languages \textbf{Iris Proof Mode}~\cite{proofmode} and \textbf{MoSeL}~\cite{mosel} for separation logic proofs in Coq, and have co-developed \textbf{Mtac2}~\cite{mtac2}, a dependently typed language for safe tactic programming in Coq.
\item Most of the aforementioned results use separate proofs to establish
  properties of programs and programming languages.
  In other work~\cite{autosound}, we have investigated how to use dependent types
  to define programming language specifications so that certain properties (like
  type safety) hold \emph{by construction}.
\end{itemize}

\section{Future research}

There are many directions for future work in formal methods. In this section
I will a (non-exhaustive) list of some directions that I plan to work on in the
coming years:

\begin{itemize}
\item I have recently been awarded an NWO \textbf{Veni} grant to apply formal
  verification to programs written in a combination of different programming languages.
  This is needed, because actual software is not written in a single programming
  language, but consist of many components written in different languages that
  interact with each other.
  As part of the Veni, I will develop formal semantics and reasoning tools
  for shared-memory interaction via foreign function interfaces,
  and message passing via sockets or signals, and apply this to the web,
  where programming language interaction is omnipresent.
\item A lot of the formal verification research has focused on functional
  correctness, \ie that the program has the correct output given some input.
  However, I think we should go beyond that, and develop sound and modular
  techniques and tools for establishing non-functional properties such as
  resource usage, complexity, security properties like non-interference, \etc
  In a recent manuscript, we have developed a separation logic to establish correct
  disposal of resources in a programming language with concurrency~\cite{iron}.
\end{itemize}

%% Bibliography
\bibliographystyle{plain}
\bibliography{bib}

\end{document}
