%-----------------------------------------------------------------------------
%
%               Template for sigplanconf LaTeX Class
%
% Name:         sigplanconf-template.tex
%
% Purpose:      A template for sigplanconf.cls, which is a LaTeX 2e class
%               file for SIGPLAN conference proceedings.
%
% Guide:        Refer to "Author's Guide to the ACM SIGPLAN Class,"
%               sigplanconf-guide.pdf
%
% Author:       Paul C. Anagnostopoulos
%               Windfall Software
%               978 371-2316
%               paul@windfall.com
%
% Created:      15 February 2005
%
%-----------------------------------------------------------------------------


\documentclass[preprint,numbers]{sigplanconf}

% The following \documentclass options may be useful:

% preprint       Remove this option only once the paper is in final form.
%  9pt           Set paper in  9-point type (instead of default 10-point)
% 11pt           Set paper in 11-point type (instead of default 10-point).
% numbers        Produce numeric citations with natbib (instead of default author/year).
% authorversion  Prepare an author version, with appropriate copyright-space text.

\usepackage{amsmath}

\newcommand{\cL}{{\cal L}}

\begin{document}

\special{papersize=8.5in,11in}
\setlength{\pdfpageheight}{\paperheight}
\setlength{\pdfpagewidth}{\paperwidth}

% For compatibility with auto-generated ACM eRights management
% instructions, the following alternate commands are also supported.
%\CopyrightYear{2016}
%\conferenceinfo{CONF'yy,}{Month d--d, 20yy, City, ST, Country}
%\isbn{978-1-nnnn-nnnn-n/yy/mm}\acmPrice{\$15.00}
%\doi{http://dx.doi.org/10.1145/nnnnnnn.nnnnnnn}

% Uncomment the publication rights used.
%\setcopyright{acmcopyright}
% \setcopyright{acmlicensed}  % default
%\setcopyright{rightsretained}

\title{Actionable Feedback during Software Development}

\authorinfo{Sebastian Erdweg}
           {Delft University of Technology}
           {}

\maketitle



\noindent
My team works on programming tools that supply developers with actionable feedback during software development. 
Feedback is actionable if it is relevant to the programmer's task, if the programmer can rely on its correctness, and if it arrives in a timely manner. 
By providing actionable feedback, we protect developers against introducing performance bottlenecks, unsafe code, security vulnerabilities, or specification violations. 
Our feedback also influences development tools such as compiler optimizations and refactorings. 
We tackle this challenging research program in two focus areas: incremental computing and practical correctness proofs.


\section{Incremental computing}

My team develops building blocks for incremental algorithms that achieve high performance when reacting to a change in their input.
Rather than repeating the entire computation over the changed input, an incremental algorithm only updates those parts of the previous result that are affected by the input change.
This way, incremental algorithms provide asymptotical speedups in theory and we have observed multiple orders of magnitude speedups in practice.

Incremental algorithms are crucial for providing actionable feedback because the feedback needs to be updated after every code change the developer makes.
Yet, existing algorithms, such as the Java type checker in Eclipse JDT, are one-off solutions that required years of engineering that cannot be reproduced.
We develop building blocks for incremental computing and collect them in frameworks that execute regular algorithms incrementally.


In the \textbf{IncA project} \cite{SzaboVE17,SzaboEV16,SzaboKBME18}, we study algorithms for incremental program analysis.
The basic idea is to store the syntax tree of programs in a relational database and to run incremental Datalog queries over these relations.
However, incremental solvers of Datalog are inherently limited in expressiveness.
In particular, they lack support for lattices, which are ubiquitous in program analysis.
We were the first to discover techniques for incrementally solving lattice-based Datalog queries and we have applied our techniques to achieve order-of-magnitudes speedups when analyzing C, Java, and Rust code.

In the \textbf{PIE project} (formerly pluto) \cite{ErdwegLW15,KonatEV18,KonatSEV18}, we develop incremental build systems.
Incremental build systems are essential for fast, reproducible software builds and enable short feedback cycles when they capture dependencies precisely and selectively execute build tasks efficiently.
A much-overlooked feature of build systems is the expressiveness of the scripting language, which directly influences the maintainability of build scripts.
We develop new incremental build algorithms that allow build engineers to use a full-fledged programming language and where task dependencies can be discovered during building.	

In the \textbf{CoCo project} \cite{KuciEBBM17,ErdwegBKKM15}, we explore novel ways for co-contextual reasoning about code and how that can be used to achieve incrementality.
Specifically, we are developing co-contextual type checkers for functional and object-oriented programming languages.
A co-contextual type checker produces context requirements rather than reading context information as it traverses a syntax tree bottom-up.
We have mostly focused on applying this technique to incremental type checking so far, yet applications to parallelization and streaming seem promising.


\section{Practical correctness proofs}
Correctness proofs ensure algorithmic results are correct.
Conversely, incorrect feedback gives developers a false sense of security that is not actually warranted by their code.
We develop theory and tools that simplify correctness proofs.
Our long-term goal is to enable analysis developers to prove correctness in little time, without requiring extensive training.
Provably correct analysis results will boost the confidence of programmers when reacting to analysis feedback.

In the \textbf{Sturdy project} \cite{KeidelB18,KeidelE17}, we explore techniques for compositional correctness proofs. 
The key idea of compositional proofs is to decompose complex verification tasks into much simpler ones. 
Developers then only need to prove the simple tasks, from which overall algorithmic correctness follows by construction. 
Key to our approach is to capture the similarities between the specification and the implementation in a single shared program, parameterized over an arrow-based interface. 
We have instantiated our technique for program analyses and proved simple analyses sound with modest effort. 
To better understand and support practical scalability, we currently apply our framework to analyses of Java and JavaScript, as well as to code generators. 

In the \textbf{Veritas project} \cite{GreweEPRM18,GreweEPM18,GreweEWM15}, we explore techniques for automated verification of complex tasks.
Specifically, we explore how existing off-the-shelve SMT solvers and first-order theorem provers can be applied to prove domain-specific verification problems.
The key idea is to translate such problems into first-order logic in a way that existing provers can support.
Due to the unpredictability of off-the-shelve solvers, this research is largely driven by empirical experiments that indicate how such translation can be successful.

In the \textbf{Soundx project} \cite{LorenzenE16,LorenzenE13}, we develop techniques to guarantee the type safety of code generators.
Code generators are hard to get right because they operate at the meta-level, where programs are data.
This makes it is easy to generate code that does not type check, which in turn is hard to debug for users since the type errors refer to generated code.
We develop automated techniques for ensuring that code generators can only produce code that is well-typed.
  









\bibliographystyle{abbrvnat}
\bibliography{mybib}




\end{document}
