\section{Concurrent Software in 10 Years}\label{sec:future}

As we have seen, over the last years, there has been enormous
progress in the area of program verification, and in particular
concerning the verification of concurrent software. By now, the theory behind
verification of concurrent software is reasonably well
understood, even though there are still open ends, but a large step
is still needed to make the results usable for all programmers, in
their every-day software
development. 

In this paper, we discussed some challenges that need to be
addressed to achieve this, and we also outlined possible
approaches to tackle them. In the coming years, we plan to develop
techniques to address these questions, which should lead to a situation where
software verification techniques will be an integral part of the software development
practice, also for highly complicated concurrent software.

When verification is an integral part of software development, developing code that is formally correct will be deemed easier than developing code without formal verification.
If correctness is built into the software compile chain, checking correctness and occasionally getting verification errors will be as commonplace as dealing with type checking errors.
In ten years, writing code without static verification might be seen as this obscure workaround that can be okay to use if you really know what you are doing.
Using automated verification will be as normal as structured programming and static type checking is now.
